%!TEX encoding=UTF-8 Unicode

% THIS IS SIGPROC-SP.TEX - VERSION 3.1
% WORKS WITH V3.2SP OF ACM_PROC_ARTICLE-SP.CLS
% APRIL 2009
%
% It is an example file showing how to use the 'acm_proc_article-sp.cls' V3.2SP
% LaTeX2e document class file for Conference Proceedings submissions.
% ----------------------------------------------------------------------------------------------------------------
% This .tex file (and associated .cls V3.2SP) *DOES NOT* produce:
%       1) The Permission Statement
%       2) The Conference (location) Info information
%       3) The Copyright Line with ACM data
%       4) Page numbering
% ---------------------------------------------------------------------------------------------------------------
% It is an example which *does* use the .bib file (from which the .bbl file
% is produced).
% REMEMBER HOWEVER: After having produced the .bbl file,
% and prior to final submission,
% you need to 'insert'  your .bbl file into your source .tex file so as to provide
% ONE 'self-contained' source file.
%
% Questions regarding SIGS should be sent to
% Adrienne Griscti ---> griscti@acm.org
%
% Questions/suggestions regarding the guidelines, .tex and .cls files, etc. to
% Gerald Murray ---> murray@hq.acm.org
%
% For tracking purposes - this is V3.1SP - APRIL 2009

\documentclass{sty/acm_proc_article-sp}





\usepackage[utf8]{inputenc}
\usepackage[english]{babel}
\usepackage[T1]{fontenc}
\usepackage{balance,tikz}

\usepackage[]{graphicx}
\graphicspath{{./img/}}

\usepackage[tight,footnotesize]{subfigure}
\usepackage[]{caption}

\usepackage{hyperref}
\hypersetup{
    % backref=true, %permet d'ajouter des liens dans...
    % % pagebackref=true,%...les bibliographies
    % colorlinks=true, %colore les liens
    % breaklinks=true, %permet le retour à la ligne dans les liens trop longs
    % urlcolor= blue, %couleur des hyperliens
    % linkcolor= black, %couleur des liens internes
    hidelinks,
    % bookmarks=true, %créé des signets pour Acrobat
bookmarksopen=true}

\usepackage{listings}
\lstset{language=C,%
    basicstyle=\ttfamily \small,%
    captionpos=b,%
    breaklines=true,
    stepnumber=1,%
    numberstyle=\scriptsize,
    numbersep=6pt,
    numbers=left,%
    tabsize=2,%
    escapechar=|,%
    xleftmargin=10pt,
    frame=tb,
    numbersep=2.5pt,%
}
%\captionsetup[lstlisting]{ margin=3pt }
\usetikzlibrary{decorations.pathreplacing,patterns}

\usepackage{multirow,booktabs,xspace,soul}

\newcommand{\TABARNAC}{\emph{TABARNAC}\xspace}
\newcommand{\Input}[1]{\input{./tex/#1}}

\usepackage{todonotes}
\newcommand{\tod}[1]{\textbf{\color{red}{\ul{TODO #1}}}}
\newcommand{\mytodo}[2]{\todo[inline, author=#1]{#2}}
\newcommand{\DB}[1]{\mytodo{David}{#1}}
\newcommand{\GH}[1]{\mytodo{Guillaume}{#1}}
\newcommand{\MD}[1]{\mytodo{Matthias}{#1}}

\begin{document}

\title{TABARNAC: Identifying and Solving Memory Access Issues on NUMA Architectures}

% You need the command \numberofauthors to handle the 'placement
% and alignment' of the authors beneath the title.
%
% For aesthetic reasons, we recommend 'three authors at a time'
% i.e. three 'name/affiliation blocks' be placed beneath the title.
%
% NOTE: You are NOT restricted in how many 'rows' of
% "name/affiliations" may appear. We just ask that you restrict
% the number of 'columns' to three.
%
% Because of the available 'opening page real-estate'
% we ask you to refrain from putting more than six authors
% (two rows with three columns) beneath the article title.
% More than six makes the first-page appear very cluttered indeed.
%
% Use the \alignauthor commands to handle the names
% and affiliations for an 'aesthetic maximum' of six authors.
% Add names, affiliations, addresses for
% the seventh etc. author(s) as the argument for the
% \additionalauthors command.
% These 'additional authors' will be output/set for you
% without further effort on your part as the last section in
% the body of your article BEFORE References or any Appendices.

\numberofauthors{4} %  in this sample file, there are a *total*
% of EIGHT authors. SIX appear on the 'first-page' (for formatting
% reasons) and the remaining two appear in the \additionalauthors section.
%
\author{
% You can go ahead and credit any number of authors here,
% e.g. one 'row of three' or two rows (consisting of one row of three
% and a second row of one, two or three).
%
% The command \alignauthor (no curly braces needed) should
% precede each author name, affiliation/snail-mail address and
% e-mail address. Additionally, tag each line of
% affiliation/address with \affaddr, and tag the
% e-mail address with \email.
%
% 1st. author
\alignauthor David Beniamine\\
       \affaddr{Univ. Grenoble Alpes, LIG, F-38000 Grenoble, France}\\
       \affaddr{CNRS, LIG, F-38000 Grenoble, France}\\
       \affaddr{Inria}\\
       \email{David.Beniamine@Imag.fr}
% 2nd. author
\alignauthor Matthias Diener\\
    \affaddr{Informatics Institute}\\
    \affaddr{UFRGS}\\
    \affaddr{Porto Alegre, Brazil}\\
    \email{mdiener@inf.ufrgs.br}
% 3rd. author
\alignauthor
Guillaume Huard\\
       \affaddr{Univ. Grenoble Alpes, LIG, F-38000 Grenoble, France}\\
       \affaddr{CNRS, LIG, F-38000 Grenoble, France}\\
       \affaddr{Inria}\\
       \email{Guillaume.Huard@Imag.fr}
\and
%% 4th. author
\alignauthor
Philippe O. A. Navaux\\
    \affaddr{Informatics Institute}\\
    \affaddr{UFRGS}\\
    \affaddr{Porto Alegre, Brazil}\\
       \email{navaux@inf.ufrgs.br}
   }
% There's nothing stopping you putting the seventh, eighth, etc.
% author on the opening page (as the 'third row') but we ask,
% for aesthetic reasons that you place these 'additional authors'
% in the \additional authors block, viz.
%\additionalauthors{Additional authors: John Smith (The Th{\o}rv{\"a}ld Group,
%email: {\texttt{jsmith@affiliation.org}}) and Julius P.~Kumquat
%(The Kumquat Consortium, email: {\texttt{jpkumquat@consortium.net}}).}
%\date{30 July 1999}
% Just remember to make sure that the TOTAL number of authors
% is the number that will appear on the first page PLUS the
% number that will appear in the \additionalauthors section.

\maketitle


\begin{abstract}

\DB{Shorten the abstract, or fix the weird spaces \ldots}
In modern parallel architectures, memory accesses represent a common bottleneck, making optimizing their behavior necessary to achieve optimal performance. This is even more important with NUMA
memories, as the access time to data depends on the memory controller that is responsible for the
access and the location of the data in the main memory. Many efforts were made to
develop adaptive tools to improve memory accesses at the runtime by optimizing the mapping of data and threads to NUMA nodes. However, the improvements provided by these tools highly depend on
the memory access pattern of the original application. A code
written without considering memory performance at all might not benefit from
them. Moreover, automatic mapping tools need time to detect the
best mapping. During this training phase, opportunities for optimizations are lost. A
deeper understanding of the memory behavior can help defining the correct mapping without the need for an analysis during execution.

In this paper, we present \TABARNAC, a set of tools and methodologies for analyzing the memory behavior of parallel applications with a focus on NUMA architectures.
The main contributions of \TABARNAC are to provide a simple yet precise analysis of the behavior, allowing the
user to easily reach a deep understanding of the memory access distribution,
as well as suggestions for the developer on how to modify the source code or perform a manual mapping to improve performance.
Using \TABARNAC, we explain why some applications do not benefit from data
and thread mapping. Moreover, we propose several simple code modifications to
improve the memory access behavior as well as manual data and thread
mappings. Finally, we evaluate the impact of these optimizations with and
without thread and data mapping, comparing our manual mapping to automated
tools.
%
\end{abstract}
% A category with the (minimum) three required fields
%\category{H.4}{Information Systems Applications}{Miscellaneous}
%%A category including the fourth, optional field follows...
%\category{D.2.8}{Software Engineering}{Metrics}[complexity measures, performance measures]
%
%\terms{Theory}
%
\keywords{Memory, Performance analysis, visualization} % NOT required for Proceedings

\Input{intro.tex}
\Input{stateOfTheArt.tex}
%\Input{motivations.tex}
\Input{design.tex}
\Input{experiments.tex}
\Input{conclusions.tex}

\bibliography{biblio,library}
\bibliographystyle{abbrv}
%\balancecolumns
\balance

\end{document}
