%!TEX encoding=UTF-8 Unicode
\section{Expriments}
\label{sec:expe}
\subsection{Setup}
\label{sec:expe-setup}
machines used, becnhmarks, tools, mechanism used \ldots
\subsection{Tool overhead}
\label{sec:expe-overhead}
give an estimated analysis time one graph
\subsection{Analysis}
\label{sec:expe-analysis}
\subsubsection{IS}
The first benchmark we studied is IS from the  NAS Parallel benchmarks OpenMp
\cite{Feng04Unstructured}. This applications computes a bucket sort using
mainly three arrays.

\begin{figure}[htb]
    \centering
    % \includegraphics{<+file+>}
    \caption{Original memory access distribution for IS}
    \label{fig:is-behaviour-orig}
\end{figure}

Figure \ref{fig:is-behaviour-orig} shows the access distributions for the three
main structures of IS class W with $4$ threads. We can see that each of these
structure have a different access pattern: \emph{key\_buff2} access
distribution shows that every threads works on a different part of the
structures which allows automated tools to do efficient data/thread mapping on
it. However \emph{key\_array} is completely shared by every threads, but the
most interesting access distribution is the one of \emph{key\_buff1}. Indeed
the access repartition seems to follow a nice Gaussian, which means a few
pages are more used than all the others. Finding a good numa balance with such
a distribution is difficult and almost impossible for automated tools.

\begin{lstlisting}[frame=single, caption=IS code responsible for the
Guassain distribution of access, label=list:is]
        m = (i > 0)? bucket_ptrs[i-1] : 0;
        for ( k = m; k < bucket_ptrs[i]; k++ )
            key_buff_ptr[key_buff_ptr2[k]]++;  
\end{lstlisting}

\begin{itemize}
    \item Identify code responsible of behaviour listing \ref{list:is}\\
        key\_buff\_ptr ==  key\_buff1 \\
        key\_buff\_ptr2 == key\_buff2 \\
        values in key\_buff2 are gaussian distributed
    \item Cyclic distribution
    \item Cyclic distrubution knowing the gaussian
    \item Modification proposed
    \item New distribution figure \ref{fig:is-behaviour-new}
    \item Experimental results
\end{itemize}

\begin{figure}[htb]
    \centering
    % \includegraphics{<+file+>}
    \caption{Memory access distribution for IS after modifications}
    \label{fig:is-behaviour-new}
\end{figure}
% \end{minipage}

Cncl:

\begin{itemize}
    \item Deep understanding of the app performance for a low cost
    \item Quick analyse with class W (few seconds)
    \item Two line of code improvements
    \item Better results with automated tools 
\end{itemize}

\subsubsection{Stream Cluster}
same methodo as above
\subsection{Results}
\label{sec:expe-results}
Hope they will be good


