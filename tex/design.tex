%!TEX encoding=UTF-8 Unicode

\section{Design}
\label{sec:design}
\subsection{Implementation}
\label{sec:design-impl}
\begin{itemize}
    \item numalize: Pin \cite{Luk05Pin}
    \item get struct info lib elf etc.
        \begin{itemize}
            \item Stack (not array in stack)
            \item Static allocs (.data section)
        \end{itemize}
    \item Replace malloc => heap
    \item Debug flags => malloc names
    \item First output 3 csv (stack (useless ?) struct (name,addr,size),
        pages)
    \item Filter smaller struct than pg\_size
    \item Dependencies:
        \begin{itemize}
            \item libelfg0: static alloc
            \item hwloc for topology vizualisation
            \item R for plots, install missing library automatically
        \end{itemize}
    \item Three execution modes:
        \begin{itemize}
            \item default; run and plot
            \item -r: run only (only dependency)
            \item -p: plot only (to customize plots)
        \end{itemize}
    \item Visu R markdown, explain main ideas
\end{itemize}
\subsection{Visualization}
\label{sec:design-visu}

Providing an easy to read visualization of memory trace is a challenge. To
avoid the user the pain to deal with huge traces, we have opted for an
assisted visualization. Once the analyze phase is done, \TABARNAC~ will
generate a html page providing a summary of the trace through several plots.
This html page is generated using R markdown, which allows us to interleave
plots with explanations. Before each plot we explain how to read it, what
usual issues it can help to understand and we give hints on how to fix them.

The visualization starts with a small introduction reminding the main
principles while developing for NUMA machines, then it shows the analysis
machine's topology using hwloc\cite{Broquedis10hwloc}.

Then the visualization focuses on data structures usage. Some structures might
be ignored for two reason: either no access have been detected during the
analysis, which happens for external libraries structures, or less than one
access over ten thousands happens on them. This is done to make the output
more readable however it is possible to ask \TABARNAC~ not to ignore
structures in the second case.

The first series of plots aims at giving informations concerning the relative
importance of data structures. It shows the sizes of each data structure and the
number reed and writes done by each thread on each structure. The global
read/write patter is also displayed. These plots gives a general idea of the
structures use by threads, it allows also to identify master/slave patterns.
Moreover knowing the read/write behaviour is very useful as it determines the
possible optimization. For instance structures written only at initialization
(or very rarely) can be (relatively) easily duplicated in a way that each NUMA
node works on a local copy.
\DB{Maybe to much details here ?}

The second series of plots is probably the most important one, it show for
each thread and each structure the distribution of accesses inside the
structure. It gives an easy way to understand data sharing between threads and
structures usage. The plots presented in sections \ref{sec:motiv-mat} and
\ref{sec:exp-analysis} comes from this section. They can be used either to
find identify inefficient memory usage or to determine the best NUMA mapping
policy.

Finally \TABARNAC provides a plot showing for each page of each structure
which thread was responsible for the first touch. This information is quite
relevant as the default policy for Linux is to map a page as close as possible
to the first thread accessing it. One thread responsible of all the first
touch of a structure is often synonym of poor performances.

\subsection{Analysis overhead}
\label{sec:expe-overhead}
\begin{itemize}
    \item NPB class B, overhead for each
    \item Can run with (relatively) small  examples
    \item No overhead at runtime
    \item Run once (or twice) optimize for each run
\end{itemize}
