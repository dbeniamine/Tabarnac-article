%!TEX encoding=UTF-8 Unicode
%!TEX root=../tabarnac.tex

\section{Conclusions and Future Work}
\label{sec:concl}

In this paper, we presented \TABARNAC: a new tool to trace, analyze and
optimize the memory behavior of parallel applications running on NUMA
machines. \TABARNAC is divided in two parts: the first is a memory tracer
based on the Pin dynamic binary instrumentation. It records for each pages the
number of read and write done by each threads. Instrumentation is the most
accurate and portable way to generate memory trace. We evaluated the cost of
the instrumentation and showed that one can expect to trace an application in
$10$ to $30$ times the original execution time.
\DB{Rewrite last sentence}
Although this overhead can seem important, as our tool analyze the general
behavior and not some performance counters, we lower it by analyzing programs
running with smaller inputs without loosing accuracy.

Presenting memory traces in an intuitive way is a challenge, to address it, we
choose to provide a summary of the trace in the form of a html page
interleaving text and plots. This html page is generate by a R-markdown
script. It start by a general explanation on NUMA machines and classical
inefficient behavior, than it show several plots providing different
interpretations of the trace. Each plots is preceded by explanations on how to
read it, what kind of issues it can help to identify and how we can usually
solve them

Using this tool, we have analyze three applications. The first is a homemade
matrix multiplication and is useful to discuss classical well known issues.
Then we analyzed \emph{IS} from the NAS Parallel Benchmarks. Finally we study
\emph{Ondes3D} a real life application which simulate seismic waves. For all
these applications \TABARNAC helps us understanding their performance issues.
Using this knowledge we have proposed some code modification resulting in
significant speedup compared to the original version and to the results
provided by automated tools.
\DB{Numbers maybe}

\DB{Future work}
Future Work:
\begin{itemize}
    \item Structure detection (fortran)
    \item Automatic detection of well known inefficient behavior
    \item \ldots
\end{itemize}

Future work will focus on two directions. First, we will improve the
structure detection support to be able to analyze Fortran programs, as many
scientific applications are written in Fortran. Second, we will work
on automatic detection of ``bad'' memory behavior, such as an all-to-all sharing,
to make the analysis even more useful.

% MD: better not mention that, Ondes3D is a real application.
% Finally, we have to analyze ``real
% life'' applications and work with developers to help them improve their
% programs.
