%!TEX encoding=UTF-8 Unicode
%!TEX root=../tabarnac.tex

\section{Conclusions and Future Work}
\label{sec:concl}


In this study, we have provided a new tool to trace and analyze the memory
behaviour of NUMA application. Although the analysis time is not negligible,
\DB{Add an approximative overhead}
it provide precise informations that can be used for several kind of
optimization. Furthermore while memory traces are usually hard to analyze as
they contains a lot of informations, \TABARNAC~ produces an easy to follow
html page interleaving trace visualization, explanation on how to interpret
them and optimization hints.
\DB{Maybe focus more on methodology}

Using this tool, we have been able to provide significant improvements and
well known benchmarks over the result obtained by automated tools.
\DB{Add some numbers / develop}

Future work will be done on three directions. First we need to improve the
structure detection to be able to analyse fortran programs as many
scientific applications are still written in fortran. Second we have to work
on automatic detection of ``bad'' memory behaviour (such as all to all sharing)
to make the analysis even easier to read. Finally, we have to analyze ``real
life'' applications and work with developers to help them improve their
programs.
