%!TEX encoding=UTF-8 Unicode
%!TEX root=../tabarnac.tex

\section{Conclusions and Future Work}
\label{sec:concl}


In this paper, we provided a new tool to trace and analyze the memory
behavior of parallel applications running on NUMA architectures. Although the analysis time is not negligible, it provide precise informations that can be used for several kinds of
optimizations. Furthermore while memory traces are usually hard to analyze as
they contains a lot of informations, \TABARNAC produces an easy to follow
output, showing trace visualizations, explanations on how to interpret
them, and optimization hints.
\DB{Maybe focus more on methodology}

Using this tool, we have been able to provide significant improvements and
well known benchmarks over the result obtained by automated tools.
\DB{Add some numbers / develop}
Future work will focus on two directions. First, we will improve the
structure detection support to be able to analyze Fortran programs, as many
scientific applications are written in Fortran. Second, we will work
on automatic detection of ``bad'' memory behavior, such as an all-to-all sharing,
to make the analysis even more useful.

% MD: better not mention that, Ondes3D is a real application.
% Finally, we have to analyze ``real
% life'' applications and work with developers to help them improve their
% programs.
