%!TEX encoding=UTF-8 Unicode
%!TEX root=../tabarnac.tex

\section{Intro}
\label{sec:intro}

Using memory is both trivial and extremely complex: any developer is able to
access it, but to use it efficiently one needs to take into account several
performance factors such as cache hierarchy, \emph{NUMA} architecture
\cite{Drepper07What} \ldots

To ease the developer's work, many efforts were made to provide automated
tools such as \emph{numactl, kMAF, NUMA balancing \ldots}\DB{more, refs}
managing memory and thread mapping on the \emph{NUMA} hierarchy. However these
tools have an overhead as they need to learn the application behavior.
Furthermore they are not able to change the memory access pattern, therefore
if the memory behavior is not designed for NUMA machines, they might not
provide any improvements. For instance if all threads are accessing data from a
page at the same time, wherever the page is mapped, it will trigger remote
accesses from all NUMA nodes but one. This kind of issue can only be solved by
modifying the memory access distribution, such optimization requires a deep
understanding of the memory behavior.

There are several tools such as VTune \cite{Reinders05VTune},
HPCToolkit\cite{Adhianto10HPCTOOLKIT} \DB{more refs? } designed to help the
developer understanding and improving its applications performances,
nevertheless most of them relies on CPU counters and provides only few
informations on the memory behavior. Tracing and the memory behavior is
complex as (almost) every instruction triggers (at least) one memory access.
Although several studies have addressed this problem using sampling
\cite{Lachaize12MemProf} or limiting the amount of informations stored
\DB{ref ??}, none of them provide a user friendly visualization or automatic hints
to help the developer improving it's application.


In this paper, we present \TABARNAC, a Tool to Analyze Behavior of
Applications Running on NUMA ArChitectures. This tool allows the user to
trace, visualize and optimize the NUMA behavior of an application.
\TABARNAC provides an easy to understand but comprehensive visualization of the trace and
provides some specific hints to improve the NUMA behavior. In Section~\ref{sec:soa}, we discuss related work, then we present some motivations in
Section~\ref{sec:motivations}. Section~\ref{sec:design} presents the design and implementation of \TABARNAC. Finally we show some example of performance improvement in Section~\ref{sec:expe} and we give our conclusions in Section~\ref{sec:concl}.
