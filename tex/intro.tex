%!TEX encoding=UTF-8 Unicode
%!TEX root=../tabarnac.tex

\section{Introduction}
\label{sec:intro}

Using memory on modern parallel shared-memory systems with a \emph{Non-Uniform Memory Access~(NUMA)} behavior is both trivial and extremely complex: an application is able to access the whole memory with the same interface, but to use it efficiently, the developer needs to take several performance factors into account, such as the cache hierarchy and the structure of the NUMA architecture~\cite{Drepper07What}.
NUMA machines are characterized by multiple memory controllers per system~\cite{Awasthi2010}, dividing the physical main memory into several NUMA nodes.
Each node can access its local memory directly, but has to transfer data
through an interconnection network to access memory on remote nodes.
Current systems usually have one memory controller per socket, but architectures with multiple controllers per socket are becoming more common~\cite{AMD2012}.
In NUMA systems, decisions about where to place the data that a parallel
application uses have a significant impact on the overall performance, with
most policies aiming at improving the \emph{locality} of memory accesses~\cite{Diener2015}.

The optimal mapping of memory pages to NUMA nodes depends on the way an application accesses the memory.
To improve the mapping without changing the application, several automatic tools were proposed~\cite{Corbet,Dashti2013,Diener2014,Piccoli2014}.
However, these tools have a runtime overhead as they need to analyze the application behavior during execution and lose opportunities for improvements during this training.
Furthermore, they are not able to change the memory access pattern for additional improvements.
Therefore, if the memory behavior is not designed for NUMA machines, their improvements might be limited.
For instance, if all threads are accessing data from a single memory
page, remote memory accesses will be triggered from all NUMA nodes but one,
wherever the page is mapped.
This kind of issue can only be solved by modifying the memory access behavior
in the source code of the application, requiring a deep understanding of this behavior.

Several tools, such as Intel's VTune~\cite{Reinders05VTune} and Performance Counter Monitor~(PCM)~\cite{Intel2012b}, the HPCToolkit~\cite{Adhianto10HPCTOOLKIT}, and AMD's CodeAnalyst~\cite{Drongowski2008}, can be used to help the
developer understand and improve the performance of parallel applications.
However, these tools rely on hardware performance counters and can therefore provide only indirect and sampled information about the memory access behavior, through cache miss statistics, for example.
Indeed tracing the memory behavior is complex, as many instructions trigger at least one memory access.
Several studies have addressed this problem using sampling
\cite{Lachaize12MemProf,McCurdy2010,Gimenez14Dissecting},
and can find out \emph{what} happens (remote access, cache miss \ldots),
\emph{where} (data structure, line of code), but not \emph{how} data structure are
accessed and shared by the different threads (which causes the remote access).

In this paper, we present \TABARNAC\footnote{\TABARNAC is available at
    \url{https://github.com/dbeniamine/Tabarnac}}, a set of \emph{Tools for
        Analyzing the Behavior of
Applications Running on NUMA ArChitectures}. \TABARNAC provides tools to trace
and visualize the memory access behavior of parallel
applications. More precisely, it helps to understand \emph{why} performance
issues occur by providing information on how data structures are accessed and
shared by the
different threads.
Since it is based on memory accesses traces, \TABARNAC has a very high
accuracy while maintaining a reasonable overhead that enables the analysis of large applications.
In an evaluation with several parallel applications, we show that relatively small code changes suggested by \TABARNAC can substantially improve application performance.

The rest of this paper is organized as follows.
In the next section, we discuss related work and compare it to our proposal.
Section~\ref{sec:design} presents the design and implementation of \TABARNAC.
Our evaluation methodology is outlined in Section~\ref{sec:metho}.
We show example analyses and performance improvements with \TABARNAC using several parallel applications in Section~\ref{sec:expe-analysis}.
Finally, we present our conclusions and discuss ideas for future work in Section~\ref{sec:concl}.
