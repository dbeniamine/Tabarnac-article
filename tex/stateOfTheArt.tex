%!TEX encoding=UTF-8 Unicode
%!TEX root=../tabarnac.tex

\section{Related Work}
\label{sec:soa}

This section presents an overview of related work in the area of data mapping mechanisms and memory access profiling of parallel applications based on shared memory.

\subsection{Data Mapping Mechanisms}
\label{sec:soa-mapping}

%From a high-level view, data mapping mechanisms can be classified into two categories.
%Mechanisms that have information about the memory accesses before the application starts executing, and mechanisms without prior information that need to determine the memory access behavior during execution of the parallel application.
There are two kinds of data mapping mechanisms: the ones which have prior
information about the memory access and the ones which do not. The first can
have higher, as the collection of information induce a potentially high overhead, limiting the amount and accuracy of information that can be collected.
But the second do not require any prior analysis.
%Furthermore, opportunities for improvements are lost while information about the memory access pattern is collected.
%On the other hand, if the memory access behavior is analyzed during application execution, no prior analysis is required.
%Figure~\ref{fig:timeline} shows a comparison of the operation of these two types of applications with a parallel application consisting of four threads.
%In this example, a mechanism with prior information can perform the mapping as soon as the parallel phase starts (or even earlier), while mechanisms without this information need to learn the behavior for some time and can perform mapping decisions only at a later stage of the execution.
%
%\begin{figure}[!b]
%    % \includegraphics[width=\linewidth]{img/timeline}
%    %!TEX encoding=UTF-8 Unicode
%!TEX root=../tabarnac.tex

\def\len{6}
\def\wid{0.3}
\def\dis{0.1}

\begin{tikzpicture}
	\footnotesize
	\node[align=left, text width=1.7cm] at (-0.5,-\wid-1.5*\dis) (label1) {\bfseries With prior information};

	\draw[thick] (0,0)              rectangle +(\len, \wid);
	\draw[thick] (1,-\wid*0-\dis*1) rectangle +(\len-1,-\wid);
	\draw[thick] (1,-\wid*1-\dis*2) rectangle +(\len-1,-\wid);
	\draw[thick] (1,-\wid*2-\dis*3) rectangle +(\len-1,-\wid);

	\draw[fill=black] (1,0)              rectangle +(0.12,\wid);
	\draw[fill=black] (1,-\wid*0-\dis*1) rectangle +(0.12,-\wid);
	\draw[fill=black] (1,-\wid*1-\dis*2) rectangle +(0.12,-\wid);
	\draw[fill=black] (1,-\wid*2-\dis*3) rectangle +(0.12,-\wid);

	\draw[very thick,densely dashed] (1,1.9*\wid) -- +(0,-\wid*5-\dis*4.3) node[near start,pos=-0.2,align=center] {Start of\\parallel phase};

	\draw [decorate,decoration={brace,amplitude=5pt}] (\len+.1,\wid) -- ++(0,-4*\wid-3*\dis) node [midway,right=2mm]{threads};

	% legend:
	\draw [] (-1.35,1.4) rectangle +(0.4,0.3) node[right,yshift=-1.4mm] { Normal execution};
	\draw [fill=black] (1.4,1.4) rectangle +(0.4,0.3)node[right,yshift=-1.6mm] { Mapping};
	\draw [fill=gray!30] (3.1,1.4) rectangle +(0.4,0.3) node[right,yshift=-1.4mm] { Training};
	\draw [pattern=crosshatch] (4.8,1.4) rectangle +(0.4,0.3) node[right,yshift=-1.4mm] { Lost opportunities};

\end{tikzpicture}


\begin{tikzpicture}
	\footnotesize
	\node[align=left, text width=2cm] at (-0.35,-\wid-1.5*\dis) (label1) {\bfseries Without prior information};

	\draw[thick] (0,0)              rectangle +(\len, \wid);
	\draw[fill=gray!30, thick] (1,0)              rectangle +(\len-1, \wid);
	\draw[fill=gray!30, thick] (1,-\wid*0-\dis*1) rectangle +(\len-1,-\wid);
	\draw[fill=gray!30, thick] (1,-\wid*1-\dis*2) rectangle +(\len-1,-\wid);
	\draw[fill=gray!30, thick] (1,-\wid*2-\dis*3) rectangle +(\len-1,-\wid);

	\draw[fill=black] (2,0)              rectangle +(0.12,\wid);
	\draw[fill=black] (4,-\wid*0-\dis*1) rectangle +(0.12,-\wid);
	\draw[fill=black] (5,-\wid*1-\dis*2) rectangle +(0.12,-\wid);
	\draw[fill=black] (3,-\wid*2-\dis*3) rectangle +(0.12,-\wid);

	\path[pattern=crosshatch] (1+0.12,0) rectangle +(1,\wid);
	\path[pattern=crosshatch] (1+0.12,-\wid*0-\dis*1) rectangle +(3,-\wid);
	\path[pattern=crosshatch] (1+0.12,-\wid*1-\dis*2) rectangle +(4,-\wid);
	\path[pattern=crosshatch] (1+0.12,-\wid*2-\dis*3) rectangle +(2,-\wid);

	\draw[very thick,densely dashed] (1,1.6*\wid) -- +(0,-\wid*5-\dis*5.5);

	\draw[decorate,decoration={brace,amplitude=5pt}] (\len+.1,\wid) -- ++(0,-4*\wid-3*\dis) node [midway,right=2mm]{threads};

	\draw[thick,|-latex] (0,-\wid*6) -- +(\len,0) node[below] {\itshape execution time};

\end{tikzpicture}

%    \caption{Comparison of data mapping mechanisms with and without prior information about the memory access behavior of a parallel application consisting of four threads. Training is similar to the normal execution, but with additional training overhead.}
%    \label{fig:timeline}
%\end{figure}

%\subsubsection{Mechanisms Without Prior Information}

%Most mechanisms that have no prior information about the memory access behavior operate within the operating system.
Traditionally, operating systems have used the \emph{first-touch}~\cite{Marchetti1995}, \emph{next-touch}~\cite{Lof2005} and \emph{interleave}~\cite{Kleen2004} policies to map memory pages to NUMA nodes.
The first-touch policy, which is the default policy in most current operating systems (such as Linux), allocates a page on the NUMA node that performs the first memory access to it.
It requires that the programmer takes care of which thread accesses data
first, or it can lead to reduced performance.
In next-touch~\cite{Lof2005}, each page is periodically migrated to the NUMA
node that performs the next access to a page. This technique is more flexible
but can lead to excessive page migrations.
The interleave policy (available in Linux via the \texttt{numactl}
tool~\cite{Kleen2004}) distributes memory pages equally among all NUMA nodes,
avoiding to stress a memory controller more than the others, but does not take any locality into account.

Newer developments in operating systems focus on refining the data mapping during the execution of parallel applications, using online profiling.
Recent versions of the Linux kernel (starting with version 3.8) contain the
NUMA Balancing technique~\cite{Corbet}, which uses page faults to determine if
a page should be migrated to a different NUMA node. To increase the accuracy
of this mechanism, extra page faults are inserted at runtime, creating an additional overhead.
%A similar proposal is the AutoNUMA approach~\cite{Corbet2012}.
%Neither mechanism maintains an access history. This eliminates the need to store the access behavior, but also makes them susceptible to excessive migrations.
%Other proposals store such an access history.
%Dashti et al.~\cite{Dashti2013} introduced the Carrefour mechanism, which uses instruction-based sampling~(IBS)~\cite{Drongowski07Instructionbased} available in recent AMD architectures~\cite{AMD2012} to detect the memory access behavior.
%kMAF~\cite{Diener2014} is a similar mechanism that uses page faults to analyze the behavior.
%Due to the access history, these mechanisms can avoid excessive migrations, but still suffer from later migrations and a runtime overhead compared to mechanisms with prior information.

% hardware
% marathe, lapt

%\subsubsection{Mechanisms With Prior Information}

Most mechanisms with prior information about memory access behavior perform
mapping on the compiler, the runtime or at a library level.
Piccoli et al.~\cite{Piccoli2014} propose a compiler extension that analyzes
the memory access pattern of parallel loops use this information to migrate
pages before executing the loop.
%Nikolopoulos et al.~\cite{Nikolopoulos2000a,Nikolopoulos2000} present an integrated compiler/OS-based data mapping mechanism based on a custom OpenMP compiler and IRIX kernel extensions. The compiler inserts instrumentation code to identify access patterns to shared memory areas and to guide migration decisions.
ForestGOMP~\cite{Broquedis2010a} requires source code annotations to identify memory access behavior and is limited to the OpenMP library.
These techniques use predictions about the memory access behavior, which might be dependent on input data and can cause wrong mappings.
%Furthermore, no improvements to the memory access pattern are performed.

Libraries such as libnuma~\cite{Kleen2004} and MAi~\cite{Ribeiro2009} provide the ability
to allocate data structures on a particular NUMA node, or with an interleave
policy. These techniques can achieve large improvements, but place the burden
of the mapping on the programmer, who has to determine the best placement
alone.
% An evolution of MAi, the Minas framework~\cite{Ribeiro2010}, optionally uses a source code preprocessor to determine data mapping policies for arrays.
%
Previous research also uses memory access traces to perform data mapping~\cite{Diener2015,Marathe2010,Bolosky1992}. These can be useful to determine the maximum gains that can be achieved with mapping policies, but are not applicable in general due to their substantial overhead and the fact that the access behavior might change with different input data and different numbers of threads, for instance.
Generic tools to evaluate parallel application performance, such as Intel's VTune~\cite{Reinders05VTune} and Performance Counter Monitor~(PCM)~\cite{Intel2012b}, the HPCToolkit~\cite{Adhianto10HPCTOOLKIT}, and AMD's CodeAnalyst~\cite{Drongowski2008}, provide only indirect information about the memory access behavior and do not propose specific strategies to improve it.


\subsection{Memory Profiling}
\label{sec:soa-profiling}

Profiling memory behavior raise two main challenges, first collecting the
information: if performance counters have been developed to get quick and
easy access to information about the CPU usage, there is no such mechanism for
the memory. Second memory access traces provides huge amounts of information
on several dimension (data structure, threads, access type (read/write),
sharing, time \ldots) displaying them to the user in a readable and meaningful
way is therefore not trivial.

\subsubsection{Data collection}

Several methods have been used to address the problem of data collection. A
lot of studies tries to deduce information from hardware performance
counters~\cite{Majo13(Mis)understanding,
Jiang14Understanding,Bosch00Rivet,Weyers14Visualization,Tao01Visualizing,DeRose01Hardware},
special registers that allow to record events such as cache misses and remote
memory accesses, among others. Still, these counters only provide a partial
view of the execution, they show events happening on the processor related to
memory, but not what triggered them. Moreover available performance counters
depends of the architecture therefore it is hard to reproduce the same
analysis on different machines with these tools.


Another approach used by several
tools~\cite{Lachaize12MemProf,McCurdy2010,Liu14Tool,Gimenez14Dissecting}
consist on  using sampling mechanisms such as AMD's Instruction Based Sampling
(IBS)~\cite{Drongowski07Instructionbased} or Intel PEBS. Not only can sampling miss important events, leading to
inaccurate characterizations, but these technologies are not portable and work
only with a few recent architecture, therefore such tools can only be used in
special circumstances.

Other studies uses hardware modification (with or without simulator)
\cite{Bao08HMTT,Martonosi92MemSpy}, although it provides efficient trace
collection it is even less portable.


Finally, instrumentation can provide these informations as in
\cite{DeRose02SIGMA,Rozar14Amelioration}, although this method is slower than
the other previously described, it is more portable and precise. Moreover as
we show in section \ref{sec:expe-overhead} an efficient instrumentation
provides an acceptable overhead.

\subsubsection{Visualization}

The second difficulty of memory analysis is to present the information in such
a way that the user can use it to improve the application. Some of the tools
previously mentioned only provide a textual
output~\cite{Lachaize12MemProf,McCurdy2010,Martonosi92MemSpy}. Even if these
tools highlight the most relevant informations, it is hard to get an overview
of the memory behavior from such output. The user might be faced with a huge
amount of information and not be able to differentiate normal behaviors from
problematic ones.


Other tools provide more advanced visualizations. For
instance, Tao et al.~\cite{Tao01Visualizing} propose a detailed view of each memory
page, showing the number of remote and local accesses from each NUMA node. Weyers et
al.~\cite{Weyers14Visualization} depict the memory bandwidth between each pair of nodes,
showing where the remote accesses occur. Other
tools~\cite{DeRose01Hardware,DeRose02SIGMA,Bosch00Rivet} provide several views
of the execution, giving the ability to correlate them with the source code as
we are used to do with traditional performance tools such as Vtune. Although
all these tools can help developers and users to understand the kind of
performance issues they are facing, they never give the reason \emph{why} the
issue is happening, or how to improve it.

Gimenez et al.~\cite{Gimenez14Dissecting} provides a NUMA oriented view of the
execution, showing which node is responsible of how many access, moreover they
provides \emph{parallel coordinate graphs} to understand where does memory
problems comes from. Still it requires a lot of experiment to the user to
understand these graphs.

Finally Liu et al.'s work~\cite{Liu14Tool} is quite near to the previous studies but
they also show some \emph{address centric} visualization which helps answering
why the performance issue occurred. However these views are not mature enough
yet and only answer a part of the question.

%Previous studies aimed to answer this question for some specific
%benchmarks~\cite{Majo13(Mis)understanding,Jiang14Understanding}.
%However, these studies use manual source code analysis and performance counters and do not provide a general tool or methodology that is usable for other applications.

\subsection{Summary of Related Work}

Summarizing our discussion of previous work in this area, we conclude that
tools to improve the memory access behavior on NUMA architectures with prior
information have the highest potential for improvements.  Several studies
provide tools to analyze the memory, they usually tells \emph{what}
performance issues occurs (cache miss, remote access), sometimes \emph{where}
it occurs (structure, function, or line of code), still they never tell
\emph{why} these events happens.  Two kinds of information can help answering
this question: what thread is responsible for the first touch (as the default
page mapping depends on it), and how are the structure accessed by the
different threads. This study provides some tools and methodology to explain
\emph{why} performance issues related to memory occurs.

%The two main challenges for this type of tool is to gather information about the memory access patterns in a fast, accurate and easy-to-use way, and providing information to the developers about the ways to improve the behavior of their applications.
%This study, provides a comprehensive solution for these challenges.

% \MD{mechanisms with prior information have highest potential for improvements, but currently:
% - lack of information about memory access behavior
% - lack of information about ways to improve behavior
% }
% Importance of Mapping:
% \begin{itemize}
%     \item First touch
%     \item Interleave
% \end{itemize}

% Why Not automated tools
% \begin{itemize}
%     \item Trainning time
%     \item Garbage in/ garbage out (matrix modulo / bloc)
% \end{itemize}

% New tool: understand why performances are bad
